% Options for packages loaded elsewhere
\PassOptionsToPackage{unicode}{hyperref}
\PassOptionsToPackage{hyphens}{url}
\documentclass[
]{article}
\usepackage{xcolor}
\usepackage[margin=1in]{geometry}
\usepackage{amsmath,amssymb}
\setcounter{secnumdepth}{-\maxdimen} % remove section numbering
\usepackage{iftex}
\ifPDFTeX
  \usepackage[T1]{fontenc}
  \usepackage[utf8]{inputenc}
  \usepackage{textcomp} % provide euro and other symbols
\else % if luatex or xetex
  \usepackage{unicode-math} % this also loads fontspec
  \defaultfontfeatures{Scale=MatchLowercase}
  \defaultfontfeatures[\rmfamily]{Ligatures=TeX,Scale=1}
\fi
\usepackage{lmodern}
\ifPDFTeX\else
  % xetex/luatex font selection
\fi
% Use upquote if available, for straight quotes in verbatim environments
\IfFileExists{upquote.sty}{\usepackage{upquote}}{}
\IfFileExists{microtype.sty}{% use microtype if available
  \usepackage[]{microtype}
  \UseMicrotypeSet[protrusion]{basicmath} % disable protrusion for tt fonts
}{}
\makeatletter
\@ifundefined{KOMAClassName}{% if non-KOMA class
  \IfFileExists{parskip.sty}{%
    \usepackage{parskip}
  }{% else
    \setlength{\parindent}{0pt}
    \setlength{\parskip}{6pt plus 2pt minus 1pt}}
}{% if KOMA class
  \KOMAoptions{parskip=half}}
\makeatother
\usepackage{longtable,booktabs,array}
\usepackage{calc} % for calculating minipage widths
% Correct order of tables after \paragraph or \subparagraph
\usepackage{etoolbox}
\makeatletter
\patchcmd\longtable{\par}{\if@noskipsec\mbox{}\fi\par}{}{}
\makeatother
% Allow footnotes in longtable head/foot
\IfFileExists{footnotehyper.sty}{\usepackage{footnotehyper}}{\usepackage{footnote}}
\makesavenoteenv{longtable}
\usepackage{graphicx}
\makeatletter
\newsavebox\pandoc@box
\newcommand*\pandocbounded[1]{% scales image to fit in text height/width
  \sbox\pandoc@box{#1}%
  \Gscale@div\@tempa{\textheight}{\dimexpr\ht\pandoc@box+\dp\pandoc@box\relax}%
  \Gscale@div\@tempb{\linewidth}{\wd\pandoc@box}%
  \ifdim\@tempb\p@<\@tempa\p@\let\@tempa\@tempb\fi% select the smaller of both
  \ifdim\@tempa\p@<\p@\scalebox{\@tempa}{\usebox\pandoc@box}%
  \else\usebox{\pandoc@box}%
  \fi%
}
% Set default figure placement to htbp
\def\fps@figure{htbp}
\makeatother
\setlength{\emergencystretch}{3em} % prevent overfull lines
\providecommand{\tightlist}{%
  \setlength{\itemsep}{0pt}\setlength{\parskip}{0pt}}
\usepackage{bookmark}
\IfFileExists{xurl.sty}{\usepackage{xurl}}{} % add URL line breaks if available
\urlstyle{same}
\hypersetup{
  pdftitle={Analyse et prévision du prix et de la volatilité du BTC-USD},
  pdfauthor={khouloud Ouni},
  hidelinks,
  pdfcreator={LaTeX via pandoc}}

\title{Analyse et prévision du prix et de la volatilité du BTC-USD}
\author{khouloud Ouni}
\date{2025-11-01}

\begin{document}
\maketitle

{
\setcounter{tocdepth}{2}
\tableofcontents
}
\subsection{1. Introduction}\label{introduction}

\subsubsection{1.1 Contexte du projet}\label{contexte-du-projet}

Le Bitcoin (BTC-USD) est un actif financier notoirement volatil, ce qui
en fait un sujet d'étude complexe mais fascinant pour la prévision de
séries temporelles. Les méthodes de prévision traditionnelles peinent
souvent à capturer ses dynamiques non linéaires, tandis que les modèles
d'apprentissage automatique et d'apprentissage profond plus récents
offrent de nouvelles possibilités.

\subsubsection{1.2 Objectif}\label{objectif}

L'objectif de ce projet est de comparer de manière exhaustive une suite
de modèles de prévision pour déterminer la méthode la plus précise pour
prédire le prix de clôture quotidien du BTC-USD. Nous comparons des
modèles économétriques classiques (ARIMAX), des modèles automatisés
(Prophet) et des modèles d'apprentissage automatique (XGBoost, LSTM).

Un objectif secondaire est de modéliser et de prévoir la volatilité (le
risque) en utilisant un modèle GARCH, qui sert également de
caractéristique d'entrée pour les modèles d'apprentissage automatique.

\subsubsection{1.3 Référence
académique}\label{ruxe9fuxe9rence-acaduxe9mique}

L'approche de ce projet, comparant les méthodes classiques aux modèles
d'apprentissage profond (comme le LSTM), s'inspire de l'analyse
documentaire de Fazel Mojtahedi et al.~(2025), qui souligne la capacité
de l'apprentissage profond à gérer les données non linéaires complexes
que l'on trouve dans les séries temporelles volatiles.

\subsection{2. Méthodologie}\label{muxe9thodologie}

\subsubsection{2.1 Acquisition et préparation des
données}\label{acquisition-et-pruxe9paration-des-donnuxe9es}

Les données quotidiennes du BTC-USD ont été téléchargées via l'API de
Yahoo Finance (\texttt{quantmod}) pour la période allant du 1er janvier
2018 à aujourd'hui. Les données brutes ont ensuite été enrichies par
ingénierie de caractéristiques (\texttt{feature\ engineering}).

\subsubsection{2.2 Ingénierie des caractéristiques (Feature
Engineering)}\label{inguxe9nierie-des-caractuxe9ristiques-feature-engineering}

Pour fournir un contexte aux modèles, les caractéristiques suivantes ont
été créées :

\begin{itemize}
\item
  \textbf{Retours Logarithmiques :} \texttt{log(close\ /\ lag(close))}
\item
  \textbf{Indicateurs techniques (via \texttt{TTR}) :}

  \begin{itemize}
  \item
    RSI (Relative Strength Index) sur 14 jours
  \item
    SMA (Simple Moving Average) sur 20 et 50 jours
  \item
    MACD (Moving Average Convergence Divergence)
  \end{itemize}
\item
  \textbf{Caractéristique de volatilité :} La volatilité ajustée (sigma)
  d'un modèle \textbf{GARCH(1,1)} entraîné sur les retours
  logarithmiques a été utilisée comme prédicteur de risque.
\end{itemize}

\subsubsection{2.3 Modèles testés}\label{moduxe8les-testuxe9s}

La comparaison a inclus cinq types de modèles :

\begin{enumerate}
\def\labelenumi{\arabic{enumi}.}
\item
  \textbf{ARIMAX :} Un modèle \texttt{auto.arima()} utilisant les
  caractéristiques techniques comme régresseurs externes (xreg).
\item
  \textbf{Prophet :} Le modèle de prévision de Facebook, également
  alimenté par les mêmes caractéristiques comme régresseurs.
\item
  \textbf{XGBoost :} Un modèle d'apprentissage automatique (Gradient
  Boosting) utilisant toutes les caractéristiques, y compris la
  volatilité GARCH.
\item
  \textbf{LSTM (Scaled) :} Un réseau de neurones récurrent (Long
  Short-Term Memory) entraîné sur les données normalisées pour gérer les
  dépendances à long terme.
\item
  \textbf{Hybrid (ARIMA-LSTM) :} Un modèle LSTM entraîné pour prédire
  les ``erreurs'' (résidus) du modèle ARIMAX.
\end{enumerate}

\subsubsection{2.4 Évaluation}\label{uxe9valuation}

Les données ont été divisées en un ensemble d'entraînement (80 \%) et un
ensemble de test (20 \%). La performance des modèles a été mesurée en
utilisant la \textbf{RMSE} (Root Mean Squared Error) et la \textbf{MAE}
(Mean Absolute Error) sur l'ensemble de test.

\subsection{3. Résultats}\label{ruxe9sultats}

\subsubsection{3.1 Performance des
modèles}\label{performance-des-moduxe8les}

Le tableau suivant résume les performances de chaque modèle sur
l'ensemble de test. Des valeurs de RMSE et de MAE plus faibles indiquent
une meilleure performance.

\begin{longtable}[]{@{}lrr@{}}
\caption{Tableau 1 : Métriques de performance des modèles (sur
l'ensemble de test)}\tabularnewline
\toprule\noalign{}
Modèle & RMSE (Erreur \$) & MAE (Erreur \$) \\
\midrule\noalign{}
\endfirsthead
\toprule\noalign{}
Modèle & RMSE (Erreur \$) & MAE (Erreur \$) \\
\midrule\noalign{}
\endhead
\bottomrule\noalign{}
\endlastfoot
ARIMAX & 3077.52 & 2395.36 \\
Prophet & 3240.63 & 2488.98 \\
XGBoost (with GARCH) & 30002.55 & 23606.90 \\
LSTM (Scaled) & 24699.67 & 20402.04 \\
Hybrid (ARIMA-LSTM) & 3244.40 & 2522.58 \\
\end{longtable}

\subsubsection{3.2 Visualisation des
performances}\label{visualisation-des-performances}

Le graphique à barres ci-dessous illustre clairement la différence de
performance. Les modèles ARIMAX, Prophet et Hybrid affichent des erreurs
faibles et comparables. Les modèles XGBoost et LSTM, tels
qu'implémentés, ont des erreurs exponentiellement plus élevées.

\begin{figure}
\includegraphics[width=50in]{output/plots/metrics_comparison_barchart} \caption{Graphique 1 : Comparaison des métriques d'erreur. Une barre plus courte est meilleure.}\label{fig:metrics_plot}
\end{figure}

\subsubsection{3.3 Comparaison des
prévisions}\label{comparaison-des-pruxe9visions}

Le graphique en facettes suivant montre les prévisions de chaque modèle
(en couleur) superposées au prix réel (en noir). Cela permet de voir
\emph{comment} chaque modèle s'est comporté.

\begin{itemize}
\item
  Les graphiques \textbf{ARIMAX} et \textbf{Prophet} montrent que les
  prévisions suivent de près le prix réel.
\item
  Les graphiques \textbf{XGBoost} et \textbf{LSTM} montrent des
  prévisions qui échouent manifestement à capturer la dynamique des
  prix, avec des échelles d'erreur très différentes.
\end{itemize}

\begin{figure}
\includegraphics[width=50in]{output/plots/price_forecast_faceted} \caption{Graphique 2 : Prévisions (couleur) vs Réalité (noir) pour chaque modèle.}\label{fig:faceted_plot}
\end{figure}

\subsubsection{3.4 Analyse des résidus
(Erreurs)}\label{analyse-des-ruxe9sidus-erreurs}

Ce graphique de diagnostic montre les erreurs de chaque modèle dans le
temps. Un bon modèle doit avoir des erreurs petites et aléatoires,
centrées autour de zéro.

\begin{itemize}
\item
  \textbf{ARIMAX, Prophet et Hybrid} montrent des erreurs contenues,
  bien qu'elles augmentent pendant les périodes de forte volatilité.
\item
  \textbf{XGBoost et LSTM} présentent des erreurs massives et
  systématiques, confirmant leur échec.
\end{itemize}

\begin{figure}
\includegraphics[width=50in]{output/plots/residuals_over_time} \caption{Graphique 3 : Erreurs du modèle (Résidus) au fil du temps.}\label{fig:residuals_plot}
\end{figure}

\subsubsection{3.5 Prévision de la volatilité
(GARCH)}\label{pruxe9vision-de-la-volatilituxe9-garch}

Enfin, le modèle GARCH(1,1) a été utilisé pour modéliser le risque. Le
graphique ci-dessous montre les retours logarithmiques réels (en gris)
et la bande de volatilité prédite par le modèle (en rouge). Le modèle a
correctement identifié les périodes de forte et de faible volatilité.

\begin{figure}
\includegraphics[width=50in]{output/plots/garch_volatility_forecast} \caption{Graphique 4 : Prévision de la volatilité du GARCH (bandes rouges) vs Retours réels (gris).}\label{fig:garch_plot}
\end{figure}

\subsection{4. Discussion et Analyse}\label{discussion-et-analyse}

Les résultats sont extrêmement clairs et fournissent deux enseignements
majeurs.

\subsubsection{4.1 Leçon 1 : L'ingénierie des caractéristiques a battu
la complexité du
modèle}\label{leuxe7on-1-linguxe9nierie-des-caractuxe9ristiques-a-battu-la-complexituxe9-du-moduxe8le}

Le modèle gagnant est l'\textbf{ARIMAX}, avec le \textbf{Prophet} en
deuxième position.

L'aspect le plus révélateur est que le modèle ARIMAX sélectionné était
un \texttt{ARIMA(0,0,0)\ avec\ erreurs}. Cela signifie que le modèle a
complètement ignoré les composantes de séries temporelles (AR, I, MA) et
a fonctionné comme un pur \textbf{modèle de régression multivariée}.

\textbf{Conclusion :} Le succès de la prévision ne venait pas d'un
modèle temporel complexe, mais de la \textbf{puissance prédictive des
caractéristiques} que nous avons créées (RSI, SMAs, MACD, volume).

\subsubsection{4.2 Leçon 2 : L'échec des ``boîtes noires'' (Black
Box)}\label{leuxe7on-2-luxe9chec-des-bouxeetes-noires-black-box}

Les modèles les plus complexes, \textbf{XGBoost} et \textbf{LSTM}, ont
échoué de manière spectaculaire.

\begin{itemize}
\item
  \textbf{Échec du XGBoost :} La raison est une erreur classique de
  préparation des données. Nous n'avons pas mis à l'échelle (scale) les
  caractéristiques pour XGBoost. Le modèle a vu la caractéristique
  \texttt{volume} (un nombre à 10 chiffres) comme étant des milliards de
  fois plus importante que le \texttt{RSI} (un nombre à 2 chiffres) ou
  la \texttt{volatilité} (ex: 0.04), et a donc ignoré ces prédicteurs
  cruciaux.
\item
  \textbf{Échec du LSTM :} Bien que nous ayons mis à l'échelle les
  données pour le LSTM, il a quand même échoué. Cela démontre une vérité
  importante : les modèles d'apprentissage profond ne sont pas magiques.
  Ils nécessitent un réglage méticuleux (tuning) des hyperparamètres
  (nombre de couches, neurones, taux d'apprentissage, etc.). Un LSTM non
  réglé est souvent moins performant qu'un modèle classique bien pensé.
\end{itemize}

\subsection{5. Conclusion}\label{conclusion}

Ce projet a réussi à identifier une méthode robuste pour prévoir le prix
du BTC-USD.

L'enseignement principal est que \textbf{l'ingénierie des
caractéristiques et la compréhension du modèle l'emportent sur la simple
complexité du modèle}. Un modèle ARIMAX bien alimenté en
caractéristiques pertinentes a surpassé des modèles d'apprentissage
profond complexes mais non réglés.

Pour des travaux futurs, l'étape évidente serait de (1) mettre à
l'échelle les données pour le XGBoost (par ex. \texttt{log(volume)}) et
(2) d'entreprendre un réglage approfondi des hyperparamètres pour le
LSTM afin de voir s'il peut alors surpasser le modèle ARIMAX.

\subsection{6. Références}\label{ruxe9fuxe9rences}

Fazel Mojtahedi, F., Yousefpour, N., Chow, S. H., \& Cassidy, M. (2025).
Deep learning for time series forecasting: Review and applications in
geotechnics and geosciences. \emph{Archives of Computational Methods in
Engineering}. \url{https://doi.org/10.1007/s11831-025-10244-5}

\end{document}
